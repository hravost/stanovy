% !TEX TS-program = pdfcslatex
% !TEX encoding = UTF-8 Unicode

% This is a simple template for a LaTeX document using the "article" class.
% See "book", "report", "letter" for other types of document.

\documentclass[10pt]{article}
\usepackage{czech}
\usepackage[utf8]{inputenc}

%%% Examples of Article customizations
% These packages are optional, depending whether you want the features they provide.
% See the LaTeX Companion or other references for full information.

%%% PAGE DIMENSIONS
\usepackage{geometry} % to change the page dimensions
\geometry{a4paper} % or letterpaper (US) or a5paper or....
% \geometry{margins=2in} % for example, change the margins to 2 inches all round
% \geometry{landscape} % set up the page for landscape
%  read geometry.pdf for detailed page layout information

\usepackage{graphicx} % support the \includegraphics command and options

% \usepackage[parfill]{parskip} % Activate to begin paragraphs with an empty line rather than an indent

%%% PACKAGES
% \usepackage{booktabs} % for much better looking tables
% \usepackage{array} % for better arrays (eg matrices) in maths
% \usepackage{paralist} % very flexible & customisable lists (eg. enumerate/itemize, etc.)
% \usepackage{verbatim} % adds environment for commenting out blocks of text & for better verbatim
% \usepackage{subfig} % make it possible to include more than one captioned figure/table in a single float
% These packages are all incorporated in the memoir class to one degree or another...

%%% HEADERS & FOOTERS
\usepackage{fancyhdr} % This should be set AFTER setting up the page geometry
\pagestyle{fancy} % options: empty , plain , fancy
\renewcommand{\headrulewidth}{0pt} % customise the layout...
\lhead{}\chead{}\rhead{}
\lfoot{}\cfoot{Strana \thepage\ z \pageref{LastPage}}\rfoot{}

\usepackage{enumerate}

%%% SECTION TITLE APPEARANCE
% \usepackage{sectsty}
% \allsectionsfont{\sffamily\mdseries\upshape} % (See the fntguide.pdf for font help)
% (This matches ConTeXt defaults)

%%% ToC (table of contents) APPEARANCE
% \usepackage[nottoc,notlof,notlot]{tocbibind} % Put the bibliography in the ToC
% \usepackage[titles,subfigure]{tocloft} % Alter the style of the Table of Contents
% \renewcommand{\cftsecfont}{\rmfamily\mdseries\upshape}
% \renewcommand{\cftsecpagefont}{\rmfamily\mdseries\upshape} % No bold!

%%% END Article customizations

%%% The "real" document content comes below...

\newcounter{clanek}
\setcounter{clanek}{0}
\newcommand{\clanek}[1]{\section*{\addtocounter{clanek}{1}\textit{článek \Roman{clanek}}\\ #1}} % customise the layout...
\newcommand{\iii}[1]{\textbf{#1}}

\newcommand{\halfspace}{\hskip 0.4ex}
\newcommand{\hravost}{Hravost o.\halfspace s.}


\title{STANOVY\\občanského sdružení\\\uv{\hravost}}
\date{} % Activate to display a given date or no date (if empty),
     % otherwise the current date is printed 

\begin{document}
\setlength{\parindent}{0pt}
\maketitle
\thispagestyle{fancy}

\begin{flushleft}

\clanek{Název a sídlo, působnost a charakter sdružení}
\begin{enumerate}
\item \iii{Název sdružení} je \uv{{\bf \hravost}} (dále jen \uv{sdružení}).
\item \iii{Sídlem sdružení} je klubovna sdružení --- adresa: Kvasiny 384, 517 02 Kvasiny
\item \iii{Sdružení působí} na celém území České republiky, sdružení se nečlení na organizační jednotky --- celé sdružení je místně příslušnou organizační jednotkou.
\item \iii{Charakter sdružení} je určen jako nepolitická organizace, nezávislá na politických stranách a orgánech státní správy a moci, ve které se sdružují fyzické i právnické osoby s~cílem propagovat deskové a karetní dovednostní hry a šířit tento druh zábavy ve~společnosti, zejména jako alternativu k méně vhodným formám trávení volného času.
\item \iii{Sdružení je} občanským sdružením založeným v~souladu se zákonem č. 83/1990 Sb., o~sdružování občanů, v~platném znění. 
\end{enumerate}

\clanek{Poslání a cíle činnosti}
\begin{enumerate}
\item \iii{Základním posláním} sdružení je rozvoj a podpora volnočasových aktivit, kulturního a sportovního vyžití včetně organizace a zajištění akcí pro členy sdružení a případné další zájemce.
\item \iii{Cíle sdružení} jsou:
\begin{enumerate}[\it a\/$_{^)}$]
\item propagace deskových a karetních her;
\item prevence kriminality mladistvých;
\item propagace elektronické podoby deskových a karetních dovednostních her;
\item šíření a vývoj elektronických podob deskových her za účelem jejich propagace.
\end{enumerate}
\end{enumerate}

\clanek{Náplň a Formy činnosti}
\begin{enumerate}
\item \iii{Náplň činnosti} musí odpovídat poslání a cílům sdružení dle článku~II. těchto stanov.
\item \iii{Hlavním formou} dosahování cílů sdružení je
\begin{enumerate}[\it a\/$_{^)}$]
\item osvětová činnost v~oblasti deskových a karetních dovednostních her;
\item pořádání turnajů v~takovýchto hrách za účelem propagace;
\item poskytování a organizace volnočasových aktivit pro děti a mladistvé v~rámci cílů sdružení;
\item vytváření materiálního zázemí pro plnění výše uvedených poslání a cílů;
\item prevence kriminality mladistvých a mladých dospělých lidí.
\end{enumerate}
\item \iii{Další formy} a konkretizaci činnosti stanoví členská schůze.
\end{enumerate}


\clanek{Členství ve sdružení}
\begin{enumerate}
\item \iii{Zakládajícím členem sdružení} je fyzická osoba, která se účastní činnosti přípravného výboru a proti jejímuž členství nebyly vzneseny odůvodněné námitky ze strany ostatních osob účastnících se činnosti přípravného výboru.
\item \iii{Členem sdružení} může být jen fyzická osoba starší 18 let nebo právnická osoba, která podá přihlášku ke členství doručenou na adresu sídla sdružení. 
\item \iii{Členství vzniká} na základě rozhodnutí členské schůze sdružení, kdy pro přijetí člena je potřeba souhlas všech přítomných členů sdružení na členské schůzi a splacení zápisného a členského příspěvku, je-li určen stanovami či rozhodnutím členské schůze.
\item \iii{Členství zaniká}
\begin{enumerate}[\it a\/$_{^)}$]
\item doručením písemného oznámení člena o~vystoupení výboru sdružení;
\item doručením písemného oznámení o~konsensu všech zbývajících členů sdružení o~vyloučení člena vylučovanému členu, například při hrubém porušení stanov sdružení;
\item smrtí fyzické osoby či zánikem právnické osoby, která je členem sdružení;
\item zánikem sdružení.
\end{enumerate}
\item \iii{Člen má právo}
\begin{enumerate}[\it a\/$_{^)}$]
\item účastnit se činnosti sdružení a jeho orgánů a být o~této činnosti informován;
\item účastnit se členské schůze, volit orgány sdružení a být do nich volen;
\item předkládat návrhy, podněty a připomínky k činnosti sdružení;
\item podílet se na stanovování cílů a forem činnosti sdružení.
\end{enumerate}
\pagebreak[2]
\item \iii{Člen má povinnost}
\begin{enumerate}[\it a\/$_{^)}$]
\item dodržovat tyto stanovy, a jednat v~souladu s cíly sdružení; 
\item v~souladu s rozhodnutím členské schůze platit členské příspěvky ve stanovené výši.
\end{enumerate}
\item \iii{Výše členského příspěvku} může být rozdílná pro různé členy sdružení v~závislosti na~jejich finančních možnostech a sociální situaci. Konkrétní výši členských příspěvků stanoví členská schůze na základě svého rozhodnutí.
\end{enumerate}


\clanek{Orgány sdružení}
\begin{enumerate}
\item \iii{Orgány sdružení} jsou:
\begin{enumerate}[\it a\/$_{^)}$]

\item členská schůze;
\item výbor a předseda sdružení;
\item revizor sdružení.
\end{enumerate}
\end{enumerate}


\clanek{Členská schůze}
\begin{enumerate}
\item \iii{Členská schůze je} nejvyšším orgánem sdružení a schází se nejméně jedenkrát ročně.
\item \iii{Členskou schůzi svolává} výbor sdružení a členská schůze je usnášeníschopná, je-li přítomna alespoň polovina členů sdružení. Nesejde-li se usnášeníschopná členská schůze, svolá výbor sdružení nejpozději do jednoho měsíce, ale ne dříve než za 10 dnů, náhradní členskou schůzi, kdy tato je usnášeníschopná bez ohledu na počet přítomných členů.
\item \iii{Členskou schůzi svolá} výbor sdružení na základě vlastního rozhodnutí v~souladu s těmito stanovami, a dále také na písemnou žádost více než $1/3$~členů sdružení a to ve lhůtě do jednoho měsíce, není-li v~žádosti uvedena lhůta pozdější.
\item \iii{Členská schůze přijímá rozhodnutí} hlasováním, kdy každý člen disponuje právě jedním hlasem; pro přijetí rozhodnutí je potřeba souhlas nadpoloviční většiny přítomných členů, pokud tyto stanovy neurčují jinak.
\item \iii{Členská schůze}
\begin{enumerate}[\it a\/$_{^)}$]
\item schvaluje stanovy sdružení a jejich změny, rozhoduje o~zániku sdružení; pro přijetí těchto rozhodnutí je potřeba konsensu všech členů sdružení;
\item volí předsedu sdružení, výbor sdružení a revizora;
\item schvaluje rozpočet, zprávu o~činnosti, zprávu o~hospodaření za minulé období a revizní zprávu;
\item přijímá nové členy;
\item rozhoduje o~vyloučení člena sdružení konsensem všech členů sdružení;
\item určuje formy a konkretizaci činnosti pro další období;
\item stanovuje výši zápisného a členských příspěvků.
\end{enumerate}
\end{enumerate}


\clanek{Výbor a předseda sdružení, jednání jménem sdružení}
\begin{enumerate}
\item \iii{Výbor sdružení je} tříčlenný, jeho funkčním obdobím je jeden rok. 
\item \iii{Výbor sdružení se skládá} z předsedy sdružení a dvou členů výboru. Činnost výboru sdružení řídí předseda sdružení, jeho hlas rozhoduje v~případě rovnosti hlasů.
\item \iii{Výbor sdružení řídí} sdružení v~období mezi členskými schůzemi a projednává věci, zprávy, změny či návrhy členů sdružení. Schůze výboru sdružení je volně přístupná členům sdružení. 
%
% \item \iii{Předseda sdružení} je statutárním orgánem sdružení. Jménem sdružení jedná navenek každý z členů výboru tak, že při podpisu jakékoliv listiny či přijetí jakéhokoliv písemného usnesení nebo rozhodnutí, je třeba podpis předsedy sdružení a minimálně jednoho dalšího člena výboru sdružení pro platnost takové listiny, usnesení či rozhodnutí, není-li v~těchto stanovách určeno jinak. v~případě projednávání věci ve formě, které nevyžaduje písemnou formu, je oprávněn jménem sdružení jednat každý ze členů výboru sdružení samostatně, není-li v~těchto stanovách určeno jinak. 
\item \iii{Předseda sdružení} je statutárním orgánem sdružení. Jménem sdružení smí jednat navenek pouze předseda sdružení, nebo předsedou písemně pověřený člen výboru sdružení, a to tak, že jeho jednání není v rozporu s usnesením členské schůze nebo rozhodnutím výboru sdružení. Listiny, písemná usnesení nebo rozhodnutí, které předseda jménem sdružení podepisuje jsou platné a pro sdružení závazné. Listiny týkající se hospodářských a finančních záležitostí sdružení jsou platné při podpisu předsedou a oběma členy výboru.
%
\item \iii{Výbor sdružení se schází} dle potřeby, zpravidla alespoň jedenkrát za 3 měsíce a k~přijetí usnesení či rozhodnutí výboru sdružení je třeba souhlas předsedy sdružení a minimálně jednoho dalšího člena výboru sdružení.
\item \iii{Výbor sdružení}
\begin{enumerate}[\it a\/$_{^)}$]
\item na nejbližší schůzi projedná věc, požádá-li o~to člen sdružení;
\item připravuje zprávu o~činnosti za minulé období;
\item připravuje návrh rozpočtu na další období;
\item připravuje případné změny stanov, návrh cílů další činnosti.
\end{enumerate}
\item \iii{Rozhodnutí} v~hospodářských a finančních věcech smí přijímat pouze výbor sdružení jako celek, a to konsensem všech jeho členů.
\end{enumerate}


\clanek{Revizor sdružení}
\begin{enumerate}
\item \iii{Kontroluje} činnost sdružení, především jeho hospodaření a plnění rozhodnutí členské schůze; jeho funkčním obdobím je jeden rok.
\item \iii{Připravuje} revizní zprávy a posudek návrhu výroční zprávy a tyto předkládá členské schůzi.
\item \iii{V~průběhu vykonávání funkce} revizora nelze takto jmenovanému členovi sdružení měnit výši členských příspěvků, vyloučit ho ze sdružení ani jiným způsobem zasahovat do~jeho činnosti ve sdružení.
\end{enumerate}


\clanek{Zásady hospodaření}
\begin{enumerate}
%
%\item \iii{Sdružení je neziskovou organizací.} Případné příjmy budou tvořit zápisné a členské poplatky, příspěvky členů, dary, dotace a granty a budou používány na činnost sdružení. Dalšími příjmy mohou být příjmy z činnosti, která je v~souladu s cíli sdružení. Sdružení může uzavřít smlouvu o~spolupráci s fyzickou nebo právnickou osobou pro finanční zajištění svých aktivit.
\item \iii{Sdružení je neziskovou organizací.} Případné příjmy budou tvořit zápisné a členské poplatky, příspěvky členů, dary, dotace a granty a budou používány na činnost sdružení. Dalšími příjmy mohou být příjmy z činnosti, která je v~souladu s cíli sdružení, zejména z~pořádání turnajů v~dovednostních karetních a deskových hrách za účelem jejich propagace.
%
%\item \iii{Výdaje sdružení} jsou zaměřeny na uskutečňování cílů sdružení v~souladu s formami činností podle těchto stanov a rozpočtem sdružení. Sdružení může v~rámci uskutečňování svých cílů najímat na některé činnosti fyzické nebo právnické osoby. Smlouvy s těmito osobami musí odsouhlasit výbor sdružení konsensem a členské sdružení na jeho nejbližší schůzi.
\item \iii{Výdaje sdružení} jsou zaměřeny na uskutečňování cílů sdružení v~souladu s formami činností podle těchto stanov a rozpočtem sdružení a na provoz sdružení.
%
\item \iii{V~případě zániku sdružení} je jeho majetek po provedené likvidaci bezplatně převeden na jinou právnickou osobu neziskového charakteru, jejíž cíle jsou blízké cílům sdružení. Jestliže nebude tato osoba nalezena do 3 měsíců od ukončení likvidace, rozdělí se zbývající majetek rovným dílem mezi členy sdružení.
\end{enumerate}


\clanek{Přechodná a závěrečná ustanovení}
\begin{enumerate}
\item \iii{V~přípravné fázi} do doby registrace sdružení a následné volby výboru sdružení na první členské schůzi jedná za klub jeho přípravný výbor, jehož členové jsou níže podepsáni.
\end{enumerate}

\vfill

\begin{flushright}
{\em v~Rychnově nad Kněžnou~~28.\halfspace května~2012}
\end{flushright}

\newcommand{\podpis}[3]{\vfill{\bf #1}\\narozen #2\\bytem #3\vskip 2pt\hrule}

\podpis{Martin Horák}{30.11.\halfspace 1976}{Kvasiny 61\\517 02 Kvasiny}
\podpis{Pavel Lang}{12.11.\halfspace 1983}{Havlíčkova 136\\516 01 Rychnov nad Kněžnou}
\podpis{Daniel Cvejn}{13.7.\halfspace 1985}{Štemberkova 846\\516 01 Rychnov nad Kněžnou}
\podpis{Michal Doležal}{27.3.\halfspace 1992}{Lipovka 113\\516 01 Rychnov nad Kněžnou}
\label{LastPage}
\end{flushleft}

\end{document}

